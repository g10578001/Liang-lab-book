\labday{Monday, 15 January 2018}

\experiment{製作git與goroot工作環境的連結}


\begin{enumerate}
\item 問題原因:

git抓下來的資料夾除了go語言程式碼之外,還有其他資料,都放到goroot裡面很雜亂。然而,go語言的程式碼只能在goroot的工作還行下執行,所以要製作連結,把程式碼連到goroot裡。

\item 解決辦法:

將git的檔案連結到goroot,goroot裡只放要執行的程式碼。
	\begin{itemize}
	\item 進到go語言執行環境資料夾
	
	\colorbox{mygray}{\$ code goroog/src/jyw}
	
	\item 使用指令ln -s <git-folder> \colorbox{yellow}{{\color{blue}<link-name>}}建立連結資料夾。
	
	{\color{blue}<link-name>:goroot裡連結資料夾的名字}
	
	\colorbox{mygray}{\$ ln -s /home/liang/class-learning/go \colorbox{yellow}{class-learning}}
	
	\item 此時jyw目錄下會增加一個「class-learning」連結資料夾,在此修改的內容會同步到git的資料夾。
			
	\end{itemize}
\end{enumerate}

\experiment{建立「RefVer」(reference version)標籤}


\begin{enumerate}
	\item 問題原因:
	
	因為要將老師之前寫的網頁蓋掉,所以先建立一個git的標籤(tag),之後需要參考時,checkout 此標籤即可。	
	
	\item 解決辦法:
	
	建立「RefVer」標籤
	\begin{itemize}
		\item 第一步驟
		
		\colorbox{mygray}{\$ git tag RefVer
			}
		
		\item 第二步驟
		
		\colorbox{mygray}{\$ git push origin RefVer}
			
	\end{itemize}
\end{enumerate}

%\experiment{製作layout.tpl}
\experiment{製作layout.tpl}

\begin{enumerate}
	\item 問題原因:
	
	每個頁面都有相同的header以及footer,製作layout.tpl樣板,可以讓每個頁面的頭尾都相同。	
	
	\item 解決辦法:
	
	\begin{itemize}
		\item 將sample.html貼到layout.tpl檔,body的部份用\{\{.LayoutContent\}\}取代。
				
		\item 修改layout.tpl中css、js、fonts…...等的路徑,改成/static/css/…….。
		
		\item 將bootstrap的css、js、fonts…...等的檔案,複製到static下對應的路徑。
		
		\item 將homepage.html中body的部份貼到index.tpl裡。
		
		\item 其他細節參考米聽紀錄106/09/29
		
		\url{https://drive.google.com/open?id=1nxYJ4HLNCTJvEDQu7bVTM1Fs-pM4SiKjOUr1xN55fYg}
		
	\end{itemize}
\end{enumerate}




